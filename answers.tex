% Use the exam document class.  Its intended for writing exam papers, but we can probably make good use of it for writing answers.
\documentclass{exam}

% Define the header and footer styles.  We specify that both a header and a
% footer should be on each page.
\pagestyle{headandfoot}
% The header has name, description of the home work and the date that the PDF
% was generated.  These are rendered in a large bold font.
\header{\bfseries\large Ben Armston}
       {\bfseries\large Math homework}
       {\bfseries\large \today}
% The first page and every other page are going to have a horizontal rule below
% the header.
\firstpageheadrule
\runningheadrule
% The footer has a single centered page count.
\footer{}{Page \thepage\ of \numpages}{}

% With the preamble complete we can start the document add optional
% introductory text and start the questions (or answers as we are using them).
\begin{document}
Some introductory text.  Many \LaTeX\ commands and environments can be used
here. There is a good chance that you won't want any introductory text; at least
not for maths homework.
\begin{questions}

% Inside the questions environment we have a \question command.  Use this to
% start a new top-level question.  It will be numbered automatically.
\question
% After the \question command write the answer. Either on the same line or
% below it. Use \( to enter a math environment and \) to exit it.
\(x = \sqrt{3}\)

\question
% Add a label so we can reference this question later.
\label{my-label}
% Questions can have parts by starting a parts environment with \begin{parts}.
% Inside that environment we have a \part command.  Again each part will be
% automatically numbered.
\begin{parts}
% Here we provide the answer on the same line as the part.  We could also have
% it on a line below.
\part \(x^2 + 4\)
\part \(x <= 0\)
\begin{subparts}
\subpart \(x < 0\) is just fine
\subpart \(x = 0\) is also fine
\end{subparts}
\end{parts}

\question
\label{another-label}
Unlike question \ref{my-label}, this question has both a direct answer and
parts.  Notice how its display differs.
\begin{parts}
\part
\begin{subparts}
\subpart \(a^2 + b^2 = c^2\)
\subpart \(e^{\pi \i} - 1 = 0\)
\subpart
\( \sum_{i=1}^{\infty} \frac{1}{n^s} = \prod_p \frac{1}{1 - p^{-s}} \)
\end{subparts}
\end{parts}


\question
\label{yet-another-label}
In question \ref{another-label} we made reference to question \ref{my-label}.
We referenced it via its label \verb!\ref{my-label}! and not by its question
number.  This is something that \LaTeX\ makes easy for us.

\end{questions}
\end{document}
